\documentclass[main.tex]{subfiles}
\begin{document}
\section{Kết quả giao diện người dùng}
\subsection{Màn hình xác thực và định danh}
\begin{itemize}
    \item Giao diện Đăng nhập / Đăng ký: Thiết kế tối giản, tập trung vào trải nghiệm người dùng với các validation form rõ ràng.
    \item Giao diện Xác thực OTP: Quy trình xác nhận email để kích hoạt tài khoản.
\end{itemize}

\subsection{Màn hình chính (Newsfeed)}
\begin{itemize}
    \item Hiển thị danh sách bài viết từ người dùng đang theo dõi và các nội dung đề xuất.
    \item Tích hợp tính năng cuộn vô hạn (Infinite Scroll) để xem nội dung liền mạch.
    \item Các nút tương tác trực quan: Thả tim, Bình luận, Chia sẻ.
\end{itemize}

\subsection{Màn hình Khám phá \& Tìm kiếm}
\begin{itemize}
    \item Grid view hiển thị hình ảnh/video thịnh hành.
    \item Thanh tìm kiếm hỗ trợ tìm theo tên người dùng hoặc Hashtag (\#).
\end{itemize}

\subsection{Màn hình Nhắn tin (Chat)}
\begin{itemize}
    \item Danh sách các cuộc hội thoại gần đây.
    \item Khung chat thời gian thực với trạng thái gửi/đã xem.
    \item Tốc độ phản hồi tin nhắn tức thì nhờ kết nối WebSocket.
\end{itemize}

\subsection{Trang cá nhân (Profile)}
\begin{itemize}
    \item Hiển thị thông tin tổng quan: Avatar, Bio, số lượng người theo dõi/đang theo dõi.
    \item Grid các bài viết cá nhân và tab bài viết đã lưu.
    \item Chức năng chỉnh sửa thông tin cá nhân.
\end{itemize}

\section{Đánh giá chức năng}
\subsection{Các kịch bản kiểm thử (Test Cases)}
\begin{itemize}
    \item \textbf{Đăng ký/Đăng nhập}: Kiểm tra tính đúng đắn của việc xử lý lỗi (sai mật khẩu, email trùng) và chuyển hướng khi thành công.
    \item \textbf{Đăng bài viết}: Đảm bảo hình ảnh được upload thành công lên Storage và dữ liệu bài viết hiển thị đúng trên Newsfeed.
    \item \textbf{Tương tác}: Kiểm tra tính đồng bộ số lượng like/comment giữa các người dùng khác nhau.
    \item \textbf{Real-time}: Kiểm tra độ trễ của tin nhắn và thông báo trên 2 thiết bị khác nhau.
\end{itemize}

\subsection{Kết quả đạt được}
Hệ thống hoạt động ổn định các chức năng cốt lõi theo đúng yêu cầu đề ra. Các lỗi logic cơ bản đã được xử lý thông qua quá trình kiểm thử và debug.


\end{document}